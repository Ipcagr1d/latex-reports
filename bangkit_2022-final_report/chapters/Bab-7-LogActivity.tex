\section*{\textit{Activity Log}}

\setstretch{1.5}
Bagian ini membahas secara detil kegiatan yang dilakukan pada saat program berlangsung. Kegiatan tentunya berdasar dari penjelasan yang telah diberikan sebelumnya. Tabel digunakan sebagai preferensi struktur \textit{activity log}, disesuaikan dengan format yang telah diberikan sebelumnya. Log tidak dibagi berdasarkan kegiatan per hari yang dilakukan namun dengan rentang seminggu yang direferensikan ke dalam jadwal resmi yang sudah dikeluarkan oleh pihak \textit{Bangkit 2022}, kolom kegiatan memuat seluruh kegiatan yang dilakukan pada rentang waktu tersebut dengan \textit{mention} kegiatan dan tidak akan membahas apa yang kegiatan tersebut bahas, tugas tertentu dalam kegiatan maupun materi yang dibahas oleh kegiatan tersebut.


\setstretch{1.5}
Hasil merupakan seperti namanya menandakan hasil yang didapat dari kegiatan yang dilakukan, hal ini dapat berupa tugas yang perlu diselesaikan atau sudah diselesaikan, proyek akhir kegiatan, maupun sertifikasi yang didapat oleh peserta. Di halaman berikutnya akan memuat tabel yang dimaksud, terdiri dari tabel-tabel yang merangkum secara singkat namun komphrehensif.


\begin{tabular}{ |p{3cm}|p{6cm}|p{3cm}|  }
 \hline
 \multicolumn{3}{|c|}{Log Kegiatan} \\
 \hline
 Minggu/Tanggal & Kegiatan & Hasil\\
 \hline
 0/7 Februari & Matrikulasi, dan kegiatan \textit{prework} & \textbf{GCP intro}, gitlab usage, \textit{programming intro} \\
 \hline
 1/14 Februari & Kursus \textit{Basic web development} di Dicoding & \textit{Static web} \\
 \hline
 2/21 Februari & \textit{Pre-read} ILT \textit{Time management}, akhir \textit{Basic web development}, mulai kursus, ILT Tech 1 \textit{JavaScript Basic} & Proyek web statis, sertifikat \textit{Basic Web Development Dicoding} \\
 \hline
 3/28 Februari & ILT \textit{Time management}, Kursus \textit{JavaScript Basic} & Essay pada \textit{time management} \\
 \hline
 4/7 Maret & Tugas \textit{time management}, ILT \textit{English Spoken Correspondence}, \textit{Pre-read Professional Branding} Akhir kursus \textit{JavaScript}, awal \textit{Backend Basic}, ILT Tech 2 & Pengumpulan tugas \textit{time management}, nilai ujian akhir kursus JavaScript disertai sertifikat dan \textit{quiz english spoken correspondence} \\
 \hline
\end{tabular}


\begin{tabular}{ |p{3cm}|p{6cm}|p{3cm}|  }
 \hline
 \multicolumn{3}{|c|}{Log Kegiatan} \\
 \hline
5/14 Maret & ILT \textit{Professional Branding \& Interview}, Akhir kursus \textit{Backend Basic}, awal kursus lab \textit{Google Cloud Computing Fundamentals} & \textit{Bookshelf API} disertai sertifikat, \textit{resume proper writings} \\
 \hline
 6/21 Maret & Tugas \textit{Professional Branding \& Interview}, \textit{Pre-read Critical Thingking}, \textit{Google Cloud Computing Fundamentals}, ILT Tech 3 & Essai \textit{Professional Branding \& interview}, \textbf{Qwiklabs badges} dan GCP \textit{stuffs} \\
 \hline
 7/28 Maret & ILT \textit{Critical Thingking}, \textbf{Qwiklab Quests} \textit{(Developer essentials, DevOps, Interactive apps, website}, awal kursus \textbf{Coursera} \textit{Architecting with Google Cloud Engine} & \textbf{Qwiklabs badges} dan sertifikat course 1 kursus \textit{Architecting with Cloud Engine} \\
 \hline
 8/4 April & Tugas \textit{Critical Thingking}, \textit{Pre-read Adaptability}, ILT Tech 4, Kursus 2 \textit{Architecting with Google Cloud Engine}, \textbf{Qwiklab quests} \textit{(Create and manage cloud resources, Foundational infrastructure task, Foundational Data, ML, and AI)} & Essai \textit{Critical Thingking}, Sertifikat kursus 2 \textit{(Architecting with GCE)}, \textbf{Qwiklab badges} \\
 \hline
 \end{tabular}

\begin{tabular}{ |p{3cm}|p{6cm}|p{3cm}|  }
 \hline
 \multicolumn{3}{|c|}{Log Kegiatan} \\
 \hline
 9/11 April & ILT \textit{Adaptability}, Kursus 3 dan 4 \textit{(Architecting with GCE), \textbf{Qwiklab quests} \textit{(Build and secure networks, Monitor and logs with Operations Suite)}} & Sertifikat 3 \& 4 \textit{(Arhitecting with GCE)}, \textbf{Qwiklab badges} \\
 \hline
 10/18 April & Tugas \textit{Adaptability}, \textit{Pre-read Idea Generation \7 MVP}, ILT Tech 5, Kursus 5 \textit{(Architecting with GCE)}, \textbf{Qwiklab quests} \textit{(Cloud Architecture, Understanding \& Optimizing Google Cloud Cost, Security \& Identity Fundamentals} & Sertifikat \textit{specialization (Architecting with GCE)}, \textbf{Qwiklab badges} \\
 \hline
 11/25 April & ILT \textit{Idea Generation \& Planning}, \textbf{Qwiklab quest} \textit{(Cloud Logging, Deploy to Kubernetes)}, Kursus \textbf{Coursera} \textit{Preparing for ACE Certification} & \textbf{Qwiklab badges}, Sertifikat \textit{Preparing for ACE Certification}. \\
 \hline
 12/9 May & Tugas \textit{Idea Generation \& MVP}, \textit{Capstone Project} & Essai \textit{Idea Generation \& MVP}, \textit{Safe Route Project Artifacts} \\
 \hline
\end{tabular}

\begin{tabular}{ |p{3cm}|p{6cm}|p{3cm}|  }
 \hline
 \multicolumn{3}{|c|}{Log Kegiatan} \\
  \hline
  13/16 May & ILT \textit{English Business Presentation}, \textit{Capstone Project} & \textit{Quiz English Business Presentation}, \textit{Phase 1 Safe Route Overview Statistics Development} \\
 \hline
 14/23 May & \textit{Capstone Project} & \textit{Phase 2 Safe Route, Routing System, Integrations}\\
 \hline
 15/30 May & \textit{Capstone Project} & \textit{Phase 3 Safe Route, Artificial System Development}\\
 \hline
 16/6 Juni & \textit{Capstone Project} & \textit{Final Phase, Deployments \& Polishing} \\
 \hline
 17/13 Juni & \textit{Capstone Project}, \textit{Pre-read Startup Valuation} & Finalisasi \textbf{Safe Route} \\
 \hline
 18/20 Juni & ILT \textit{Startup Valuation}, \textit{English Post-test}, awal \textbf{Dicoding} \textit{Cloud Certification simulation exam} & Tugas \textit{Startup Valuation} \\
 \hline
 19/27 Juni & \textit{Pre-read Professional Communication}, akhir \textbf{Dicoding} \textit{Cloud Certification simulation exam} & Preparasi \textbf{Google ACE} \\
 \hline
  20/4 Juli & ILT \textit{Professional Communication}, Kelas \textit{Expert} (Opsional) & Essai \textit{Professional Communication} \\
  \hline
  21/11 July & \textit{Post Program} & Program selesai \\
  \hline
\end{tabular}