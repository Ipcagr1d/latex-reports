{\setstretch{1.5}
Ucapan terima kasih saya berikan kepada \textbf{Kementrian Pendidikan, Kebudayaan, Riset dam Teknologi Republik Indonesia} yang telah melalui kebijakan \textit{Merdeka Belajar Kampus Merdeka}, dengan salah satu programnya yaitu \textbf{Bangkit 2022} yang telah memberikan kesempatan yang bisa dibilang cukup langka, yaitu kesempatan untuk mempelajari teknologi terbaru dengan kurikulum yang disesuaikan dengan dunia \textit{professional}. Tidak lupa juga puji syukur kepada Tuhan Yang Maha Esa yang telah memberikan kesempatan tersebut kepada saya, serta ucapan terima kasih terutama kepada:
\begin{enumerate}
    \item Ibu Prof. Dr. E.S. Margianti, SE., MM., selaku Rektor Universitas Gunadarma
    \item Ibu Dr. Lintang Yuniar Banowosari, S.Kom., M.Sc., selaku Ketua Jurusan Teknik
Informatika Universitas Gunadarma.
    \item Ibu Hurnaningsih, S.Kom., MM., selaku dosen pendamping dalam program Bangkit 2022.
    \item Ibu Ike Putri Kusumawijaya, ST., MMSI., selaku dosen pendamping dalam program Bangkit 2022.
    \item Ibu Astie Darmayantie, ST, MMSI, MSc., beserta rekan yang membantu memfasilitasi mahasiswa Gunadarma dalam mendaftar ke dalam program Bangkit 2022.
    \item Seluruh tim Bangkit 2022 beserta jajaran pimpinan, atas usahanya dalam membentuk lingkungan belajar yang menyenangkan dan mendukung perkembangan siswa.
    \item Seluruh mentor yang telah berkontribusi dalam Bangkit 2022.
\end{enumerate}
Tidak lupa juga, ucapan rasa terima kasih terhadap semua yang telah mewujudkan program \textbf{Bangkit 2022} ini yang terlalu banyak untuk disebutkan satu-satu. Semoga ilmu yang didapatkan dengan terlaksananya program ini dapat turut digunakan dalam rangka memajukan bangsa Indonesia, Berikut adalah laporan akhir yang merupakan sebagai media visibilitas dari pengaruh program \textbf{Bangkit 2022} ini.

\vspace{2.3cm}

\begin{flushright}
Karawang, 20 Juli 2022 \\
\vspace{2cm}
I Putu Cahya Adi Ganesha
\end{flushright}
}
