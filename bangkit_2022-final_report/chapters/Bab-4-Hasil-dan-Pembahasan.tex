Pembelajaran dalam program \textbf{Bangkit 2022} dirasa cukup untuk dapat membantu mengatasi krisis talenta digital yang sedang dihadapi Indonesia, dibuktikan dengan materi yang komprehensif pada tiap \textit{learning path}, dan keberhasilan pada pengembangan aplikasi proyek akhir yaitu \textbf{Safe Route} yang menjadi testimoni kualitas program dan sebagai proyek yang menjadi \textit{highlight} dari laporan ini yang berperan sebagai sebuah aplikasi disertai dengan sistem yang diharapkan dapat mengurangi tingkat kriminalitas di Indonesia khusunya di kota-kota besar.

\section{Kesimpulan}

Program \textbf{Bangkit 2022} memiliki materi yang cukup komphrehensif untuk masing-masing \textit{path}, diikuti dengan pengembangan karakter siswanya. Pembelajaran yang meliputi:
\begin{itemize}
    \item Pengembangan Web
    \item \textit{Cloud Computing Platform}
\end{itemize}
Yang kemudian diimplementasikan langsung ke dalam proyek akhir (\textit{capstone project} menjadi testimoni kualitas program. Sedangkan bila proyek akhir disimpulkan akan diperoleh poin-poin sebagai berikut:
\begin{itemize}
    \item Berhasilnya pengembangan aplikasi \textbf{Safe Route} dengan kolaborasi dari ketiga unsur \textit{learning path} yang ada di \textbf{Bangkit 2022}.
    \item Sistem eksperimental pencegahan kriminalitas secara preventif dan reaktif yang ditenagai kecerdasan artifisial dan \textit{cloud computing platform \textbf{GCP}}.
    \item Aplikasi yang dibuat secara \textit{native} dengan \textbf{Kotlin}.
\end{itemize}

\section{Saran}

Terdapat beberapa hal yang dapat diperbaiki dalam rangka membantu dan mengembangkan kemajuan program ke depannya hal ini adalah:
\begin{itemize}
    \item Ilmu komputer dasar yang mencakup jaringan, sistem operasi, dan Linux dirasa perlu dilakukan.
    \item Transparansi sistem penilaian.
    \item Perbaikan beberapa sistem seperti absensi dan sistem \textit{feedback} yang kurang maksimal.
    \item Ketepatan waktu jadwal, terjadi banyak keterlambatan di akhir.
\end{itemize}
Sedangkan bila berbicara mengenai proyek akhir yang dikerjakan, yaitu \textbf{Safe Route} berikut adalah saran-saran yang ditunjukkan kepada siapapun yang ingin ikut menyempurnakan aplikasi serta sistemnya:
\begin{itemize}
    \item \textit{Business model} perlu diperkuat, karena pada fase yang ada sekarang tidak banyak kesempatan untuk mendapatkan keuntungan.
    \item \textit{ETL pipeline(s)} yang lebih matang.
    \item Serta data riil yang dapat merefleksikan kejadian kejahatan secara akurat.
\end{itemize}