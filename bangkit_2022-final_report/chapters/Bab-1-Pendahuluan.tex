\section{Latar Belakang}

Revolusi digital yang sedang dialami oleh Indonesia, yang salah satu bukti riilnya yaitu gaungan \textit{Revolusi Industri 4.0} yang sering didengar belakangan ini menuntut perkembangan yang sangat besar pada talenta digital Indonesia, yang sayang-nya sampai saat ini sedang mengalami \textit{"krisis talenta digital"}. Krisis tersebut merujuk kurangnya sumber daya manusia profesional yang bergerak pada bidang digital, sedangkan ketersediaan tersebut harus segera dipenuhi karena tuntutan yang turut berkembang secara eksponensial. 

Tiga dari banyak sektor yang memiliki permintaan yang paling banyak adalah \textit{mobile developer}, mengacu kepada penggunaan perangkat \textit{smartphone} dalam kehidupan sehari-hari yang semakin hari semakin banyak penggunanya sehingga menjadikannya \textit{prime market} untuk pengembang, \textit{cloud computing} yang merujuk kepada adopsi besar-besaran \textit{cloud platform} di dunia guna mendukung kebutuhan \textit{developer} maupun perusahaan dari skala kecil ke besar yang dalam penggunaannya sendiri membutuhkan \textit{expertise} tersendiri, dan \textit{machine learning} yang merujuk kepada penggunaan kecerdasan artifisial yang makin sering digunakan untuk mengotomatisasi berbagai tugas sederhana ke komplex sehingga membutuhkan seseorang yang ahli untuk mewujudkan \textit{model} untuk mengimplementasikannya. 

Program \textbf{Bangkit 2022}, merupakan program yang sesuai kutipan dalam laman resminya \textit{"Dirancang untuk mempersiapkan siswa dengan keterampilan yang dibutuhkan dan sertifikasi teknologi, kurikulum Bangkit menawarkan 3 jalur pembelajaran interdisipliner - pembelajaran mesin, pengembangan seluler, dan komputasi awan. Pada akhir program ini, Anda akan dilengkapi dengan keahlian teknologi dan soft skill yang Anda butuhkan untuk berpindah dari dunia akademis ke tempat kerja dan sukses di perusahaan terkemuka."} dan \textit{Tahun ini, Bangkit ditawarkan sebagai program Studi Independen Bersertifikat Kampus Merdeka yang disetujui dan didukung oleh Kementerian Pendidikan dan Kebudayaan Republik Indonesia. Kami mendaftarkan hingga 3.000 mahasiswa di 3 jalur pembelajaran untuk membantu mereka mengembangkan keterampilan yang dibutuhkan di bidang teknologi dan mempersiapkan mereka untuk mengikuti sertifikasi Google.Program ini berlangsung sepanjang semester genap 2022.} Bertujuan sebagai program yang mempersiapkan siswasnya agar \textit{job-ready}, dengan \textit{skill} yang sedang memiliki banyak sekali permintaan, seperti pengembangan \textit{mobile}, \textit{cloud computing}, dan \textit{machine learning}.

\section{Lingkup} \label{lingkup}

Dalam program ini saya berperan sebagai \textit{mentees/cohort} atau siswa \textit{cloud computing} mempelajari mengenai penggunaan \textit{cloud platform Google}, yang perlu melakukan beberapa kegiatan (akan dibahas lebih detail pada bab 2). Dengan fokus berada pada pengimplementasian dari ilmu yang telah didapat dari program ini dituangkan dalam sebuah proyek akhir (\textit{capstone project}), yang menjadi salah satu bahan utama dalam evaluasi peserta \textbf{Bangkit 2022} dalam mengaplikasikan ilmu yang telah didapat ke dalam sebuah produk yang siap diluncurkan dan diikuti dengan tema yang cukup menarik tentunya. Proyek ini menuntut seluruh aspek dari \textit{learning path} terwujud, mulai dari basis proyek yang dilakukan dengan \textit{kotlin} yang dirancang khusus untuk \textit{android}, penambahan \textit{core feature} oleh \textit{machine learning} dan infrastruktur pendukung dari \textit{cloud computing}.

\section{Tujuan}

Dengan mengambil visi dan misi dari program \textbf{Bangkit 2022} ini, tujuan pribadi saya dalam mengikuti program ini tentunya mencari ilmu pengetahuan yang dapat diandalkan serta \textit{up-to-date} dengan \textit{demand} yang dapat ditemukan secara riil pada kehidupan sehari-hari, dalam jangka pelaksanaan program, pembelajaran mengenai \textit{web development} secara fundamental yang terdiri dari \textit{frontend} dan \textit{backend}, diikuti dengan pembelajaran yang berkaitan dengan \textit{cloud} itu sendiri dalam \textbf{Google Cloud Platform}, yang di dalamnya dapat dijabarkan menjadi sub-material seperti jaringan, sistem operasi, \textbf{DevOps}, \textbf{SRE}, basis data dan lain-lain.

Serta dapat mengimplementasikan hal-hal yang telah disebutkan di atas ke dalam sebuah produk jadi yang diharapkan dapat berperan sebagai solusi terhadap masalah yang dialami oleh penduduk di Indonesia. Pada akhirnya deretan acara dalam program mendorong siswanya untuk mempelajari sebuah keahlian yang sedang dalam permintaan, fasih dalam keahlian tersebut agar dalam status yang \textit{work-ready} diikuti dengan \textit{soft-skill} yang akan turut membantu kehidupan para siswanya.

Mendasari pernyataan ini berdasarkan pernyataan yang sudah diberikan sebelumnya pada latar belakang dan lingkup, secara \textit{Long-term} tujuan utama dari pengadaan program ini adalah untuk mengatasi \textit{"krisis talenta digital"}, yang diharapkan dapat mendongkrak perkembangan Indonesia.